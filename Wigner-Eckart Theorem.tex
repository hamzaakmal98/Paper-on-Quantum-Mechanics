\documentclass[a4paper,11pt]{article}
\usepackage{jheppub} % for details on the use of the package, please see the JINST-author-manual
\usepackage{lineno}
\usepackage{amsmath}
\usepackage{tensor}
\usepackage{mathtools}
\usepackage{physics}
\DeclarePairedDelimiter\bra{\langle}{\rvert}
\DeclarePairedDelimiter\ket{\lvert}{\rangle}
\DeclarePairedDelimiterX\braket[2]{\langle}{\rangle}{#1 \delimsize\vert #2}


\title{\boldmath The Wigner-Eckart Theorem: Proof, Applications in Perturbation Theory, and Beyond}


\providecommand{\keywords}[1]
{
  \small	
  \textbf{\textit{KEYWORDS:---}} #1
}

\author[1]{H. Akmal}
\emailAdd{24100232@lums.edu.pk}
\affiliation{PHY-312, Department of Physics, SBASSE, LUMS.}


\abstract{The Wigner-Eckart theorem stands as a cornerstone in quantum mechanics, providing invaluable insights into the matrix elements of tensor operators within rotationally invariant systems. This paper aims to present a comprehensive analysis of the Wigner-Eckart theorem, beginning by encompassing its proof, applications in perturbation theory, and later exploring related content including but not limited to Angular Momentum and Rotation Generators, Rotational Symmetry, Selection Rules, and applications of the Wigner-Eckart theorem in Molecular Spectroscopy and Graph Theory.
}

\keywords{Wigner-Eckart, Perturbation, Rotation}

\begin{document}
\maketitle
\flushbottom
\pagenumbering{arabic}
\newpage
\section{Understanding the Wigner-Eckart Theorem - an intuition}
\label{sec:intro}


The report begins by elucidating the fundamental principles of the Wigner-Eckart theorem, introducing the mathematical formalism and its historical development. The theorem's proof is presented, highlighting the factorization of matrix elements into reduced matrix elements and Clebsch-Gordan coefficients. By establishing this theorem, a powerful tool for calculating transition probabilities and understanding selection rules in quantum mechanical systems is unveiled.
\\

The subsequent focus of the report lies in exploring the applications of the Wigner-Eckart theorem within the realm of perturbation theory. Perturbation theory plays a crucial role in understanding the behavior of quantum systems subject to small perturbations, and the theorem's implications in this context are highly significant. By leveraging the theorem, we can investigate how perturbations affect the matrix elements and transition probabilities of operators, shedding light on the dynamics and properties of quantum systems under external influences.
\\

Moreover, the research paper encourages an exploration of additional content related to the Wigner-Eckart theorem. This may involve delving into its applications in spectroscopy, atomic and molecular multipole moments, and the representation theory of groups. By expanding the scope of the study, a more comprehensive understanding of the theorem and its implications in various fields of quantum physics can be achieved.  By doing so, it strives to provide researchers and students with a comprehensive resource to comprehend the theorem's fundamental concepts, its significance in perturbation theory, and its broader applications in quantum mechanics.
\\

\section{Angular Momentum}

In quantum mechanics, angular momentum is a quantized observable associated with the rotational symmetry of a system. It arises due to the non-commutativity of position and momentum operators. The angular momentum operator, denoted as 
$\hat{L}$, consists of three components: $\hat{L}_z$, $\hat{L}_z$, and $\hat{L}_z$, corresponding to rotations around the x, y, and z axes, respectively. These components satisfy the commutation relations [$\hat{L}_x$,$\hat{L}_y$] = i\hbar $\mathcal{$\varepsilon$}_{ijk}$$\hat{L}_z$, where  $\mathcal{$\varepsilon$}_{ijk}$ is the Levi-Civita symbol.
\\

The total angular momentum of a system is represented by the operator $\hat{J}$, which can be written as a sum of the orbital angular momentum operator $\hat{L}$ and the intrinsic spin angular momentum operator $\hat{L}$. The eigenvalues of $\hat{J}^2$ and $\hat{J}_z$ provide the magnitude and projection of the total angular momentum, respectively. The eigenfunctions of the angular momentum operators are spherical harmonics, which possess a specific pattern of angular variation.
\\

Perturbation theory is a powerful tool for analyzing the behavior of quantum systems under small perturbations. It provides a systematic approach to calculating corrections to the eigenstates and eigenvalues of a system's Hamiltonian when subjected to a perturbation. In perturbation theory, the angular momentum plays a vital role in the expansion of the perturbed wavefunctions and energies. The eigenstates and eigenvalues of the unperturbed system serve as a basis for constructing the perturbed states. The perturbation affects the system's angular momentum, causing changes in the matrix elements of angular momentum operators and leading to corrections in the energy levels and transition probabilities.
\\

The Wigner-Eckart theorem establishes a fundamental connection between the matrix elements of angular momentum operators and the Clebsch-Gordan coefficients. It provides a powerful tool for calculating transition probabilities and selection rules in quantum mechanical systems.
\\

The theorem states that the matrix elements of a tensor operator, such as angular momentum, can be factorized into a reduced matrix element and a product of Clebsch-Gordan coefficients. This factorization simplifies the calculation of matrix elements and facilitates the analysis of angular momentum transitions within a rotationally invariant system.
\\

In perturbation theory, the Wigner-Eckart theorem enables the determination of matrix elements of angular momentum operators in the presence of small perturbations. By factoring out the reduced matrix elements, the theorem allows for the systematic calculation of corrections to the matrix elements, thereby aiding in the understanding of perturbed angular momentum transitions and their associated probabilities.
\\

Angular momentum is a central concept in quantum mechanics, governing the rotational symmetry and behavior of physical systems. Its relationship with perturbation theory and the Wigner-Eckart theorem provides a comprehensive framework for analyzing the effects of perturbations on angular momentum transitions and calculating transition probabilities.
\\

By understanding the quantization of angular momentum, utilizing perturbation theory, and applying the Wigner-Eckart theorem, physicists can explore a wide range of phenomena, from atomic and molecular spectroscopy to nuclear and particle physics. 
\\


\section{Rotation Operator}
Rotational symmetry is a fundamental concept in quantum mechanics that plays a crucial role in understanding the behavior of physical systems. It stems from the invariance of physical laws under rotations, and it has profound implications for the quantization of angular momentum, selection rules, and the description of quantum states.In quantum mechanics, the rotational symmetry operator, denoted as $R(\theta)$, represents a rotation by an angle θ around a specific axis. The rotation operator acts on the wavefunctions that describe the quantum states of a system. It is a unitary operator, ensuring that the normalization and orthogonality of the wavefunctions are preserved.
\\

Mathematically, the rotation operator can be expressed as $R(\theta) = e^{-i\hat{J}\theta/\hbar}$, where $\hat{J}$ is the angular momentum operator and $\hbar$ is the reduced Planck constant. This exponential form arises from the commutation relations of the angular momentum operators, [$\hat{J}_x$,$\hat{J}_y$] = i\hbar $\mathcal{$\varepsilon$}_{ijk}$$\hat{J}_z$, where  $\mathcal{$\varepsilon$}_{ijk}$ is the Levi-Civita symbol.. The rotation operator provides a mathematical representation of the transformation of quantum states under rotations.
\\

The eigenstates of angular momentum correspond to specific values of $\hat{J}^2$ and $\hat{J}_z$. These eigenstates are determined by solving the eigenvalue equation: 

\begin{equation}
\label{eq:EigenValues of J^2}
\begin{aligned}
\hat{J}^2 \ket{j,m} = \hbar^2j(j + 1) \ket{j, m}. 
\end{aligned}
\end{equation}

\begin{center}
   and
\end{center}
\begin{equation}
\label{eq:EigenValues of Jz}
\begin{aligned}
\hat{J_z}\ket{j,m} = \hbar m \ket{j, m},
\end{aligned}
\end{equation}


where j is the total angular momentum quantum number and m is the magnetic quantum number. The eigenstates are often represented using the spherical harmonics, which exhibit a specific pattern of angular variation.
\\




\section{Selection Rules in Rotational Symmetry}

Rotational symmetry imposes selection rules on the transitions between quantum states. These selection rules arise from the conservation of angular momentum during transitions. For a quantum system, the total angular momentum $\hat{J}$ is conserved, meaning that the initial and final states must have the same total angular momentum.
\\

The selection rules dictate that the change in the z-component of angular momentum,

\begin{equation}
\label{eq:Selection Rules on m}
\begin{aligned}
\Delta m = m_f - m_i,,
\end{aligned}
\end{equation}

must be an integer. This restriction leads to a discrete set of allowed transitions between quantum states. Furthermore, the magnitude of $\Delta m$ is limited by the values of $j_i$ and $j_f$, leading to the well-known selection rule 

\begin{equation}
\label{eq:Selection Rules on j}
\begin{aligned}
\hat|\Delta m|\leq \Delta j, \Delta j = |j_f - j_i|.,
\end{aligned}
\end{equation}

\section{Perturbation Theory}
Perturbation theory is a powerful tool in quantum mechanics that allows us to analyze the behavior of quantum systems when subjected to small perturbations. It provides a systematic approach to calculating corrections to the eigenstates and eigenvalues of a system's Hamiltonian, leading to a better understanding of the effects of external influences on quantum systems. This paper explores the concept of perturbation theory in quantum mechanics, its mathematical framework, and its applications in various fields of study.
\\

Perturbation theory is employed when the Hamiltonian of a system can be separated into two parts: $\hat{H_o}$, representing the unperturbed system whose eigenstates and eigenvalues are known, and $\hat{V}$, representing a small perturbation that affects the system. The perturbation parameter, denoted as $\lambda$, quantifies the strength of the perturbation.
\\

The idea behind perturbation theory is to treat $\hat{V}$ as a small correction to $\hat{H_o}$. This allows us to expand the wavefunctions and energies of the perturbed system in terms of the eigenstates and eigenvalues of the unperturbed system. The perturbed wavefunctions and energies are expressed as power series in $\lambda$, with higher-order terms representing higher-order corrections.
\\

First-order perturbation theory provides an initial approximation to the perturbed wavefunctions and energies. The first-order correction to the energy is given by:
\\

\begin{equation}
\label{eq:First Order Correction to energy}
\begin{aligned}
\hat{E_1} = \expval{V}{\Psi}.,
\end{aligned}
\end{equation}

where $\ket{\Psi_n}$ represents the nth eigenstate of the unperturbed system and $\bra{\Psi_n}$ is its corresponding bra vector. This correction accounts for the effect of the perturbation on the energy levels of the system.
\\

The first-order correction to the wavefunction is given by:

\begin{equation}
\label{eq:First Order Correction to the Wavefunction}
\begin{aligned}
\ket{\Psi_n} = \sum_{k=1}^{\infty} k \neq n \bra{\Psi_k} V \ket{\Psi_n} / (\hat{E_n} - \hat{E_k})\ket{\Psi_k},
\end{aligned}
\end{equation}
where the sum extends over all eigenstates $\ket{\Psi_k}$ except the nth state. This correction describes the mixing of different eigenstates due to the presence of the perturbation.
\\
The second-order correction to the wavefunction can be expressed as a sum of terms involving intermediate states:
This correction takes into account the effects of virtual transitions to intermediate states and their subsequent interaction with the perturbation.
\\

\section{The Wigner-Eckart Theorem}
The Wigner-Eckart theorem is a fundamental result in quantum mechanics that relates the matrix elements of operators within a rotationally invariant system. It provides a powerful tool for calculating transition probabilities and selection rules in quantum mechanical systems.
\\

The theorem is named after Eugene Wigner and Carl Eckart, who independently derived it in the 1920s. It is primarily used in the context of angular momentum theory and the representation theory of groups.
\\

The Wigner-Eckart theorem states that the matrix element of a tensor operator between states of definite angular momentum can be factorized into a reduced matrix element and a product of Clebsch-Gordan coefficients. Mathematically, it can be expressed as follows:

\begin{equation}
\label{eq:Wigner Eckart Theorem}
\begin{aligned}
\bra{\alpha ', j', m'} {T^k}_q \ket{\alpha , j , m} = \bra{\alpha ', j'} |T^k|\ket{\alpha , j}   \braket{j,k;m,q}{j,k;j'm'},
\end{aligned}
\end{equation}
where $\alpha$ and $\alpha _0$ are other quantum numbers of the problem. The first factor is known as the reduced matrix element. 
\\

We now give an intuitive proof of the theorem.
\\

Since the states are eigenstates of angular momentum operators, we have:

\begin{equation}
\label{eq: Application on states}
\begin{aligned}
\bra{k,q}  \hat{J}_z  \ket{k,q} = \hbar q \braket{k,q'}{k,q},
\end{aligned}
\end{equation}
from which it follows:

\begin{equation}
\label{eq:Jz action}
\begin{aligned}
\sum_{q'=-k} ^{+k} {T^k}_q' \bra{k,q}  \hat{J}_z  \ket{k,q} = \hbar q \braket{k,q'}{k,q} = \hbar q {T^k}_q = [\hat{J}_z,{T^k}_q ],
\end{aligned}
\end{equation}


The matrix elements of $\hat{J_{+_-}}$ are represented as:

\begin{equation}
\label{eq:Matrix Elements of J+-}
\begin{aligned}
\bra{k,q'} \hat{J_{+_-}} \ket{k,q}, = \hbar \sqrt{(k \pm q)(k \pm q +1)} \delta _{q',q' \pm 1}
\end{aligned}
\end{equation}
which when acted on by the same operation gives 

\begin{equation}
\label{eq:Action of Operator}
\begin{aligned}
\sum_{q'=-k} ^{+k} {T^k}_q'  \bra{k,q'} \hat{J_{+_-}}\ket{k,q} = [\hat{J_{+_-}}, {T^k}_q ],
\end{aligned}
\end{equation}
Now using the Wigner- Eckart Theorem, lets find matrix elements of arbitrary sets of states:


\begin{equation}
\label{eq:Matrix elements of arbitrary sets of states}
\begin{aligned}
\bra{\alpha', j', m'} [\hat{J}_z,{T^k}_q ] -\hbar {T^k}_q \ket{\alpha, j, m} = 0,
\end{aligned}
\end{equation}
\\\\

States on both sides are eigenstates of $\hat{J}_z$ therefore, 

\begin{equation}
\label{eq:States are eigenstates of Jz}
\begin{aligned}
(m' - m - q)   \bra{\alpha', j', m'} {T^k}_q \ket{\alpha, j, m} = 0,
\end{aligned}
\end{equation}


Expanding the commutator and using the operator $ \hat{J_{+_-}}$ on the states gives us:

\begin{equation}
\label{eq:Recursion Relation for WE}
\begin{aligned}
\begin{center}
    \sqrt{(j'\pm m')(j' \pm m'+ 1)} \bra{\alpha', j', m\pm 1'} {T^k}_q \ket{\alpha, j, m}  = \sqrt{(k\pm q)(k \pm q + 1)} \bra{\alpha ', j',m'} |{T^k}_(q\pm 1) \ket{\alpha,j,m}} +  \sqrt{(j\pm m)(j \pm m+ 1)} \bra{\alpha ', j',m }{T^k}_q \ket{\alpha,j,m\pm 1}},
\end{center}

\end{aligned}
\end{equation}

which is identical to the recursion relation between the Clebsch Gordan coefficients derived in the study of addition of angular momentum reproduced below:


\begin{equation}
\label{eq:Recursion Relation for CG}
\begin{aligned}
\begin{center}
    $\sqrt{(j\pm m)(j\pm m +1)} \braket{j_1,j_2;j,m\pm 1}{j_1,j_2;m_1,m_2}$ =
    $\sqrt{(j_1\pm m_1)(j_1\pm m_1 +1)} \braket{j_1,j_2;j,m\pm 1}{j_1,j_2;m_1 \pm 1,m_2}$ =
    $\sqrt{(j_2\pm m_2)(j_2\pm m_2 +1)} \braket{j_1,j_2;j,m\pm 1}{j_1,j_2;m_1,m_2\pm 1}$.
\end{center}

\end{aligned}
\end{equation}

\\\\
\begin{table}[htbp]
\centering
\begin{tabular}{r|c}
\hline
Wigner Eckart Application Relation & Clebsch Gordan Recursion Relation\\
\hline
j' & j\\
j & $j_1$ \\
k & $j_2$ \\
m' & m \\
m & $m_1$ \\
q & $m_2$ \\
\hline
\end{tabular}
\caption{Translation of Coefficients from W-E (Eq 6.8)to C-G coefficients (Eq 6.9) .\label{tab:i}}
\end{table}

Using the reality of the Clebsch Gordan coefficients to interchange the order of representations achieves the desired results. The equations are identical if one realizes that the action of ${T^k}_q$ on the ket is equivalent to adding to $\ket{j,m}$ state a state with angular momentum $\ket{k,q}$. The equivalence becomes very clear when the following substitutions are made:


Using the completeness of the recursion relations derived from Wigner Eckart and Clebsch Gordan, we can eradicate the sum and obtain a translational relationship between the 2 coefficients which can then be solved for the ratio of any 2 quantities as follows:

\begin{equation}
\label{eq:Coefficients ratio}
\begin{aligned}
   a_{i,j} y_j = 0.
    \\
    \frac{x_i}{x_j} = \frac{y_i}{y_j} = c.,
\end{aligned}
\end{equation}
where c is a proportionality factor and so the matrix elements of ${T^k}_q$ between two angular momentum states becomes proportional to the Clebsch Gordan Coefficients and the derived constant of proportionality c in (Eq 6.10) which is not dependent on m, m' or q'.
\\
This completes the proof of the theorem.

\section{Applications of the Wigner Eckart Theorem alongside Perturbation Theory and Next Steps}
\\
\subsection{Quantum Field Theory}
The Wigner-Eckart theorem, originally developed in the context of atomic physics, finds significant applications in quantum field theory as well. This theorem provides a powerful tool for calculating matrix elements of tensor operators and is particularly useful in understanding the selection rules and symmetry properties of quantum systems.


In quantum field theory, tensor operators arise when studying the interactions between particles. These operators describe the transformation properties of the fields under certain symmetry operations, such as rotations or Lorentz transformations. The Wigner-Eckart theorem allows us to decompose the matrix elements of these tensor operators in terms of reduced matrix elements and Clebsch-Gordan coefficients, simplifying their evaluation.


In quantum field theory, the Wigner-Eckart theorem finds applications in various contexts. One important application is in the study of electromagnetic transitions between quantum states. The electromagnetic field is represented by the vector potential, which is a tensor operator. By applying the Wigner-Eckart theorem, one can calculate the transition probabilities and selection rules for electromagnetic transitions, providing insights into the decay and scattering processes in quantum field theory.


Another application of the Wigner-Eckart theorem in quantum field theory is in the analysis of weak interactions. Weak interactions involve the exchange of gauge bosons, such as the W and Z bosons, and are described by the electroweak theory. The Wigner-Eckart theorem allows for the calculation of weak interaction amplitudes and matrix elements, leading to a deeper understanding of processes involving weak interactions, such as beta decay and neutrino interactions.


Furthermore, the Wigner-Eckart theorem is also utilized in the study of nuclear physics within the framework of quantum field theory. Nuclear interactions can be described by effective field theories, where nuclear operators are represented as tensor operators. The Wigner-Eckart theorem aids in the analysis of nuclear transitions, the calculation of nuclear matrix elements, and the determination of nuclear structure and properties.

Perturbation theory plays a central role in quantum field theory, which provides the theoretical framework for studying the interactions between elementary particles. Quantum field theories involve the quantization of fields, such as the electromagnetic field or the Higgs field, and the description of particle interactions through Feynman diagrams.

Perturbation theory in quantum field theory enables the calculation of scattering amplitudes and cross-sections for particle interactions. It involves expanding the perturbative series in terms of coupling constants, such as the fine structure constant or the strong interaction coupling constant. Perturbation theory provides a systematic way to calculate higher-order corrections to processes involving the exchange of virtual particles.


By utilizing the Wigner-Eckart theorem, researchers can extract valuable information about symmetries, selection rules, and interaction strengths, deepening our understanding of the fundamental processes and properties of quantum field systems.

\subsection{Atomic and Molecular Spectroscopy}

The Wigner-Eckart theorem is a valuable tool in molecular spectroscopy, aiding in the analysis and interpretation of spectroscopic data. It provides a framework for understanding the selection rules and intensity patterns observed in molecular spectra, offering insights into molecular structure and properties.

In molecular spectroscopy, the interaction of molecules with electromagnetic radiation leads to the absorption, emission, or scattering of photons. The transitions responsible for these spectral features are governed by selection rules, which determine the allowed changes in molecular states. The Wigner-Eckart theorem plays a central role in establishing these selection rules and calculating transition probabilities.

By applying the Wigner-Eckart theorem, the matrix elements of molecular transition operators can be decomposed into reduced matrix elements and Clebsch-Gordan coefficients. These matrix elements represent the transition amplitudes between molecular states and are essential for determining the intensities of spectral lines.

The theorem simplifies the calculation of transition probabilities and intensities by factorizing the matrix elements. The reduced matrix elements depend only on the total angular momentum quantum numbers of the initial and final states, independent of the specific molecular basis set used. This property allows for a more efficient analysis of spectroscopic data, providing a systematic approach to understanding molecular symmetries.

In rotational spectroscopy, the Wigner-Eckart theorem helps determine selection rules for transitions between rotational energy levels. It enables the calculation of transition probabilities, providing insights into molecular geometry and rotational dynamics.

In vibrational spectroscopy, the theorem assists in understanding selection rules for transitions between vibrational energy levels. It allows for the calculation of transition intensities, aiding in the characterization of molecular vibrational modes and potential energy surfaces.

Similarly, in electronic spectroscopy, the Wigner-Eckart theorem helps determine selection rules for transitions between electronic energy levels. It enables the calculation of transition probabilities, providing insights into electronic structure and properties.

Note also that Rotational symmetry plays a vital role in the analysis of atomic and molecular spectra. Spectroscopic transitions involve the absorption or emission of photons by atoms or molecules, resulting in changes in their quantum states. The selection rules imposed by rotational symmetry determine the allowed transitions and the corresponding spectral lines.

In atomic spectroscopy, the quantization of angular momentum leads to the formation of spectral series, such as the Balmer series in hydrogen, characterized by specific sets of energy levels and transition wavelengths. The selection rules determine which transitions are allowed and which are forbidden, providing insights into the electronic structure and energy levels of atoms.


\subsection{Solid State Physics}
In solid-state physics, perturbation theory is widely employed to study the behavior of electrons in a crystalline lattice. The electronic structure of solids can be described by the tight-binding model, which includes the unperturbed Hamiltonian representing the isolated atoms and the perturbation arising from the crystal lattice.

Perturbation theory allows for the calculation of corrections to the electronic band structure, energy levels, and transport properties of solids. It helps in understanding phenomena such as bandgap renormalization, phonon-electron interactions, and the effects of impurities and defects on the electronic properties of materials.

\subsection{Quantum Optics}
Perturbation theory finds wide-ranging applications in quantum optics, the study of the interaction between light and matter. It allows for the analysis of the effects of external electromagnetic fields, such as lasers, on the behavior of atoms, molecules, and other quantum systems.

Perturbation theory in quantum optics enables the calculation of transition probabilities, absorption and emission rates, and the determination of coherent and incoherent scattering processes. It provides a framework for understanding phenomena such as spontaneous emission, Raman scattering, and the behavior of two-level systems interacting with quantized electromagnetic fields.

Many experimental techniques in quantum optics, such as laser spectroscopy, quantum information processing, and quantum communication, heavily rely on perturbation theory to design and interpret experiments accurately.

\subsection{Symmetry Analysis}
The Wigner-Eckart theorem provides a powerful tool for analyzing the symmetries of physical systems. By decomposing tensor operators into reduced matrix elements and Clebsch-Gordan coefficients, the theorem allows us to identify the symmetries present in a system and determine their effect on the matrix elements. This information is crucial in understanding the behavior of physical systems under symmetry operations, leading to insights into conservation laws and fundamental properties.

\subsection{Group Theory}
The Wigner-Eckart theorem connects the concept of conservation laws with symmetries. In quantum mechanics, conservation laws arise from symmetries of physical systems. The theorem provides a mathematical framework to relate these symmetries to the matrix elements of tensor operators and, consequently, to the conservation laws associated with those operators. This connection deepens our understanding of fundamental principles, such as the conservation of angular momentum and the conservation of parity.

The Wigner-Eckart theorem is closely tied to the field of group theory, which deals with the mathematical study of symmetries. The theorem's formulation and application rely on the group-theoretic properties of angular momentum operators and their transformation properties. Consequently, the theorem contributes to the broader field of group theory, enabling the exploration and understanding of group representations, character tables, and related concepts.

\appendix
\section{Acknowledgments}








% Bibliography

%% [A] Recommended: using JHEP.bst file
%% \bibliographystyle{JHEP}
%% \bibliography{biblio.bib}

%% or
%% [B] Manual formatting (see below)
%% (i) We suggest to always provide author, title and journal data or doi:
%% in short all the informations that clearly identify a document.
%% (ii) please avoid comments such as "For a review'', "For some examples",
%% "and references therein" or move them in the text. In general, please leave only references in the bibliography and move all
%% accessory text in footnotes.
%% (iii) Also, please have only one work for each \bibitem.

\begin{thebibliography}{99}

\bibitem{g}
Nouredine Zettili,
\emph{Quantum Mechanics, Concepts and Application},
\emph{John Wiley & Sons} {\bf vol 1} (2001) Ch 7.4.4

\bibitem{b}
Cotton, F. A. ,
\emph{Chemical applications of group theory},
Wiley (1971).

\bibitem{c}
Levine, R. D.,
\emph{. Molecular reaction dynamics.},
Cambridge University Press. (2019).

\bibitem{a}
Herzberg, G.,
\emph{Molecular spectra and molecular structure},
\emph{Spectra of diatomic molecules} {\bf Van Nostrand 1} (1950) 

\bibitem{d}
AuWigner, E., & Eckart, C.thor,
\emph{Über die Symmetrien der Wellenfunktionen. Zeitschrift für Physik},
 59(7-8), 549-557.

\bibitem{e}
Pimentel, G. C., & McClellan, A. L. ,
\emph{The hydrogen bond.},
WH Freeman (1960).


\end{thebibliography}
\end{document}
